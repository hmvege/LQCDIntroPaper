\documentclass[11pt]{article}

\usepackage[utf8]{inputenc}
\usepackage{amsmath}
\usepackage{mathtools}
\usepackage{amsfonts}
% For figures and graphics'n stuff
\usepackage{graphicx}
\usepackage{caption}
\usepackage{subcaption}
\usepackage{cite}
% \usepackage[numbers]{natbib}
\usepackage{url}
\usepackage{color}
\usepackage{float}
\usepackage{hyperref}
\usepackage{relsize}
% \usepackage{xspace}
\usepackage{bm}
\usepackage{tabularx}
\usepackage[toc,page]{appendix}
\usepackage{enumerate}

\usepackage{algorithm}
\usepackage{algorithmicx}
\usepackage{algpseudocode}


% \overfullrule=2cm

\newcommand{\husk}[1]{\color{red} #1 \color{black}}
\newcommand{\expect}[1]{\left\langle{#1}\right\rangle}
\newcommand{\bra}[1]{\langle{#1}|}
\newcommand{\ket}[1]{|{#1}\rangle}
\newcommand{\rn}{\mathbf{r}^\text{new}}
\newcommand{\ro}{\mathbf{r}^\text{old}}

\newcommand{\CC}{C\nolinebreak\hspace{-.05em}\raisebox{.4ex}{\tiny\bf +}\nolinebreak\hspace{-.10em}\raisebox{.4ex}{\tiny\bf +}}
\def\CC{{C\nolinebreak[4]\hspace{-.05em}\raisebox{.4ex}{\tiny\bf ++}}}

\title{An introduction to Lattice Quantum Chromodynamics}
\author{Mathias M. Vege, Giovanni Pederiva}

\date{\today}
\begin{document}
\maketitle

\begin{abstract}

\end{abstract}

% \tableofcontents

\section{Introduction}
The aim of this paper is to give a hands-on introduction to the how one can go from a rudimentary understanding of quantum mechanics and quantum field theory, to simulating quantum chromodynamics on the lattice. In order to kick this is of, let us begin by discussing our end-goal, Quantum Quantum Chromodynamics(QCD).

QCD is the theory for interacting quarks and gluons. It has since its inception gotten a reputation notoriously difficult to work with, much because it is not linear due to its three- and four-vertex gluon self interactions. This makes solving it rather difficult, and in our search for understanding this theory we come to embrace lattice QCD.

The Lattice QCD machinery is at its core a application of the Metropolis algorithm on solving path integrals. LQCD may at first glance seem familiar to the much simpler Ising model, but in contrast to the Ising model there are many more tools which need to be mastered and understood. Questions such as how to create a discretized field theory, how to analyze sparse datasets or parallelize code requiring communication between nodes will all be addressed in due time. The first step will be to understand the path integral formalism. Since this is a quantum theory, we will refresh ourselves on some basic quantum mechanics. 

If the reader feels so inclined, feel free to skip to chapter about the Path Integral Formalism.

\section{Refreshing quantum mechanics}
% Fundamental postulates of quantum mechanics.
\subsection{The postulates of quantum mechanics}
Let us begin by recapitulating the postulates of quantum mechanics. Depending on what literature you are looking at, a different number of postulates will be listed [R Shankar, Griffiths, Laurent Nottale]. However, let us start by looking at those which count as the most fundamental ones, and expand upon them\footnote{If one wants to, there are actually three subsets of postulates in QM[see laurent nottale], the fundamental ones, the secondary ones which can be derived from the fundamental ones, and the principles which are more consequences of the fundamental ones.}.

\begin{enumerate}[I]
	\item \textit{Complex state function.} The state of a quantum mechanical system can be described by a complex valued vector, $\ket{\psi}$ that lives in Hilbert space $\mathcal{H}$. Vectors in Hilbert space is often referred to as states.
	\item \textit{The correspondence principle.} A physical observable has an Hermitian operator. For every dynamical variable in classical mechanics, there is a corresponding quantum mechanical one. Further, the principle states we reproduce the classical results in the limit of large quantum numbers. Further, on order to convert to from classical to quantum mechanics, we replace the Poisson brackets with the the commutator ones divided by $i\hbar$.
	\begin{align}
		\{\cdot,\cdot\} \rightarrow \frac{1}{i\hbar}[\cdot,\cdot]
		\label{eq:poisson}
	\end{align}
	Further, any observable $\hat{\Omega}$ measured will result in a eigenvalue $\omega$ associated with the observable. These are related through the equation eigenvalue equation, $\hat{\Omega}\ket{\psi}=\omega\ket{\psi}$
	\item \textit{Von Neumann's postulate(collapse of state).} If we measure a system $\ket{\psi}$ with an observable $\Lambda$ and get the eigenvalue of $\omega_i$, then the system will be in a state $\ket{\psi_i}$ after the measurement\footnote{A measurement is here an ideal experiment which we minimally disturb the system and it is in compliance with theory. In classical mechanics, an ideal measurement is a measurement in which the system remains unaffected.}. We say that the system has \textit{collapsed} into state $\ket{\psi_i}$.
	\item \textit{The Schröedinger equation.} We require a state vector $\ket{\psi(t)}$ to abide by the Schröedinger equation,
	\begin{align}
		i\hbar\frac{d}{dt}\ket{\psi(t)} = \hat{H} \ket{\psi(t)}
		\label{eq:schroedinger}
	\end{align}
	where the Hamiltonian $\hat{H}$ is a linear Hermitian operator constructed accordingly to the correspondence principle.
	\item \textit{Born's postulate.} The wave function squared $|\psi|^2$ is interpreted as a the probability of the system at a given configuration. For instance, the Probability of getting the eigenvalue $\omega_i$ in a state $\ket{\psi}$ is given by $|\langle\omega_i|\psi\rangle|^2$. This implies that the wave function must be normalized,
	\begin{align}
		\langle \psi | \psi \rangle = 1
		\label{eq:wf-normalized}
	\end{align}
\end{enumerate}
From these postulates follow several other properties and consequences. The first being expectation values.

\subsubsection{Expectation values}
Given some observable $\hat{\Omega}$ and a system described by a normalized wave function $\psi$, we have that the expectation value of that is given as the mean value from statistics.
\begin{align*}
	\langle \Omega \rangle &= \sum_i P(\omega_i) \omega_i = \sum_i |\bra{\omega_i}\psi\rangle|^2\omega_i \\
	&= \sum_i \bra{\psi}\omega_i\rangle \langle\omega_i \ket{\psi} \omega_i \\
	&= \sum_i \bra{\psi} \omega_i \ket{\omega_i} \langle\omega_i \ket{\psi} \\
	&= \sum_i \bra{\psi} \hat{\Omega} \ket{\omega_i} \langle\omega_i \ket{\psi} \\
	&= \bra{\psi} \hat{\Omega} \cdot I \ket{\psi} \\
\end{align*}
We used that $I = \sum_i \ket{\omega_i}\bra{\omega_i}$ and that the $\omega_i\ket{{\omega_i}} = \hat{\Omega}\ket{\omega_i}$. This gives us the expression for the expectation value,

\begin{align}
	\expect{\Omega} = \bra{\psi} \hat{\Omega} \ket{\psi}
	\label{eq:expectation-value}
\end{align}
INCLUDE DIFFERENCE BETWEEN PROB. AMP. AND PROB. DENSITY(mostly coz Mathias easily confuses this kinda stuff and needs to repeat it for himself).

\subsubsection{Superposition principle}
Applying the superposition principle in quantum mechanics, we get from the linearity of the Hamiltonian $\hat{H}$ in the Schröedinger equation\eqref{eq:schroedinger} that a quantum mechanical state is built up from a set of linear independent states. E.g.,
\begin{align*}
	\ket{w} = a\ket{u} + b\ket{v}
\end{align*}

\subsubsection{Expansion in eigenfunctions}
A state $\ket{\psi}$ can be expanded in a basis of eigenfunctions $\ket{\psi_n}$ as
\begin{align}
	\psi = \sum_n c_n \ket{\psi_n}
	\label{eq:eigenfunction-expansion}
\end{align}
We demand orthogonality form this set, such that 
\begin{align}
	\bra{\psi_n}\psi_m\rangle = \delta_{nm}
	\label{eq:eig-func-orthogonality}
\end{align}

% Straight onto bra-ket formalism and what they mean
\textit{IS NEXT SECTION NEEDED?}
Let us blast of with recapping some of the formalism of quantum mechanics, in particular what that is needed to perform calculations we are interested in.

Technical notes:
difference between probability amplitude and probability density.

\section{The path integral formalism}
The path integral formalism saw its complete form in 1948 with the help of Richard Feynman\cite{RevModPhys.20.367}(but was however conceptualized somewhat earlier). The concept is both intuitive and elegant, and forms the basis of modern quantum field theory.

\subsection{Path integral formalism in quantum mechanics}
As the path integral formalism was first applied to quantum mechanics, it serves as a natural starting point. In quantum mechanics one usually starts by finding the eigenvalues and eigenstates of the Hamiltonian $H$, then proceeds to find the propagator $U(t)$. The basic idea of a path integral is to go directly to expressing the propagator $U(x,t;x_0,t_0)$. This propagator is describes the dynamics of a state $x_0$ evolving into another state $x$. Informally, we may think if this as:

\begin{enumerate}
	\item Draw all paths connecting $(x_0,t_0)$ and $(x,t)$.
	\item Find the action(which serves as a weight) $S[x(t)]$ for each path $x(t)$.
	\item The propagator can now be expressed as a sum of all paths written in an informal way,
	\begin{align*}
		U(x,t;x_0,t_0) = A \sum_\text{all paths} e^{iS[x(t)]/\hbar},
	\end{align*}
	with $A$ as a normalization constant.
\end{enumerate}

\begin{align}
	U(x_1,t_1=t;x_0,t_0=0) \equiv U(x_1,x_0,t) = \bra{x_1} \exp \left( -\frac{i}{\hbar} H t \right) \ket{x_0}
	\label{eq:propagator-definition-qm}
\end{align}

\subsection{Definition of the propagator}

\subsection{Euclidean path integrals}

\subsection{Correlation functions}

\subsection{Wick rotations} % Move up?

\subsection{The Euclidean picture}

\subsection{Observables for the path integral}

\subsection{Switching between the Heisenberg and Schröedinger picture.}



% Some exercises? Do some myself?

% go beteween Heisenberg and Schröedinger picture



\section{The Metropolis algorithm}
\input{metropolis/metropolis}

\section{Statistical analyses}
What is autocorrelation? Autocorrelation versus correlation?
\subsection{Bootstrapping}
\subsection{Jackknife method}
\subsection{Blocking}


\section{Quantum Field Theory and its fundamentals}
random citaton from peskin \cite{peskin}
\subsection{Observables}
\subsection{Action}


\section{Lattice Quantum Chromodynamics}
\subsection{Making a theory gauge invariant}
\subsection{The Plaquette}
\subsection{The Wilson gauge action}
\subsection{Notes on a Lattice QCD simulation}
\subsubsection{Updating matrices}
\subsubsection{Generating random \texorpdfstring{$SU(3)$}{SU3} matrices}
\subsubsection{Generating random \texorpdfstring{$SU(2)$}{SU2} matrices}


\bibliographystyle{unsrt}
\bibliography{bibliography/LQCDIntro}

\end{document}
