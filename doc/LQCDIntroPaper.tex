\documentclass[11pt]{article}

\usepackage[utf8]{inputenc}
\usepackage{amsmath}
\usepackage{mathtools}
\usepackage{amsfonts}
% For figures and graphics'n stuff
\usepackage{graphicx}
\usepackage{caption}
\usepackage{subcaption}
\usepackage{cite}
% \usepackage[numbers]{natbib}
\usepackage{url}
\usepackage{color}
\usepackage{float}
\usepackage{hyperref}
\usepackage{relsize}
% \usepackage{xspace}
\usepackage{bm}
\usepackage{tabularx}
\usepackage[toc,page]{appendix}

\usepackage{algorithm}
\usepackage{algorithmicx}
\usepackage{algpseudocode}


% \overfullrule=2cm

\newcommand{\husk}[1]{\color{red} #1 \color{black}}
\newcommand{\expect}[1]{\left\langle{#1}\right\rangle}
\newcommand{\bra}[1]{\langle{#1}|}
\newcommand{\ket}[1]{|{#1}\rangle}
\newcommand{\rn}{\mathbf{r}^\text{new}}
\newcommand{\ro}{\mathbf{r}^\text{old}}

\newcommand{\CC}{C\nolinebreak\hspace{-.05em}\raisebox{.4ex}{\tiny\bf +}\nolinebreak\hspace{-.10em}\raisebox{.4ex}{\tiny\bf +}}
\def\CC{{C\nolinebreak[4]\hspace{-.05em}\raisebox{.4ex}{\tiny\bf ++}}}

\title{An introduction to Lattice Quantum Chromodynamics}
\author{Mathias M. Vege}

\date{\today}
\begin{document}
\maketitle

\begin{abstract}

\end{abstract}

% \tableofcontents

\section{Introduction}
The aim of this paper is to give a hand-on introduction to the how one can go from a rudimentary understanding of quantum mechanics and quantum field theory, to simulating quantum chromodynamics on the lattice. In order to kick this is of, let us begin by discussing our end-goal, Quantum Quantum Chromodynamics(QCD).

QCD is the theory for interacting quarks and gluons. It has since its inception gotten a reputation notoriously difficult to work with, much because it is not linear due to its three- and four-vertex gluon self interactions. 

\section{Refreshing quantum mechanics}
\section{The path integral formalism}
Wick rotation.
\section{The Metropolis algorithm}
\section{Statistical analyses}
What is autocorrelation? Autocorrelation versus correlation?
\subsection{Bootstrapping}
\subsection{Jackknife method}
\subsection{Blocking}
\section{Quantum Field Theory and its fundamentals}
\subsection{Observables}
\subsection{Action}
\section{Lattice Quantum Chromodynamics}
\subsection{Making a theory gauge invariant}
\subsection{The Plaquette}
\subsection{The Wilson gauge action}
\subsection{Notes on a Lattice QCD simulation}
\subsubsection{Updating matrices}
\subsubsection{Generating random \texorpdfstring{$SU(3)$}{SU3} matrices}
\subsubsection{Generating random \texorpdfstring{$SU(2)$}{SU2} matrices}


% \bibliography{refs}{}
% %\bibliographystyle{plain}
% \bibliographystyle{plainnat}

\end{document}