The path integral formalism saw its complete form in 1948 with the help of Richard Feynman\cite{RevModPhys.20.367}(but was however conceptualized somewhat earlier). The concept is both intuitive and elegant, and forms the basis of modern quantum field theory.

\subsection{Path integral formalism in quantum mechanics}
As the path integral formalism was first applied to quantum mechanics, it serves as a natural starting point. In quantum mechanics one usually starts by finding the eigenvalues and eigenstates of the Hamiltonian $H$, then proceeds to find the propagator $U(t)$. The basic idea of a path integral is to go directly to expressing the propagator $U(x,t;x_0,t_0)$. This propagator is describes the dynamics of a state $x_0$ evolving into another state $x$. Informally, we may think if this as:

\begin{enumerate}
	\item Draw all paths connecting $(x_0,t_0)$ and $(x,t)$.
	\item Find the action(which serves as a weight) $S[x(t)]$ for each path $x(t)$.
	\item The propagator can now be expressed as a sum of all paths written in an informal way,
	\begin{align*}
		U(x,t;x_0,t_0) = A \sum_\text{all paths} e^{iS[x(t)]/\hbar},
	\end{align*}
	with $A$ as a normalization constant.
\end{enumerate}

\begin{align}
	U(x_1,t_1=t;x_0,t_0=0) \equiv U(x_1,x_0,t) = \bra{x_1} \exp \left( -\frac{i}{\hbar} H t \right) \ket{x_0}
	\label{eq:propagator-definition-qm}
\end{align}

\subsection{Definition of the propagator}

\subsection{Euclidean path integrals}

\subsection{Correlation functions}

\subsection{Wick rotations} % Move up?

\subsection{The Euclidean picture}

\subsection{Observables for the path integral}

\subsection{Switching between the Heisenberg and Schröedinger picture.}



% Some exercises? Do some myself?

% go beteween Heisenberg and Schröedinger picture

